% $Id: CoralDB-API.tex,v 1.20 2008-03-03 21:39:27 beringer Exp $

\begingroup
\newcommand{\code}[1]{\texttt{#1}}
\newcommand{\args}[1]{\textsl{#1}}

%\subsection{Interfaces and Tools}

\subsection{C++ API}

% Main author: Andrei

The main application programming interface (API) for CoralDB is the one written in C++ that is described in this section.
{\bf For the pixel detector, it has been agreed that all write access (i.e.\ all
database access for writing, updating or deleting)
should be exclusively performed through this C++ API. }

For reading the contents of the
connectivity database in certain applications, other programming languages may be more convenient.
In the interest of maintainability and in view of possible future database schema changes, access from
other languages should always be made through a well-defined interface that is made part of the CoralDB
code base.

By default CoralDB caches the advisory ``locked'' status of database
tags.  It checks whether a tag is locked when the tag is set as
``current'', and then uses that information until either a different
tag is chosen, or the \code{updateCache()} method is called, or the
tag is locked in the database via \emph{this} CoralDB instance.
\textbf{Stale cache data can allow to modify a tag that was locked by
  another client.}  The use of this cache can be completely disabled by
calling the \code{setCacheUse()} method.  The performance penalty is
an additional database query for each method that modifies the
database (in many cases this means two queries instead of one).

%% ================================================================
\subsubsection{Database authentication}

\code{CoralDB} delegates handling of database authentication to the
LCG \code{CORAL} library.   A possible way of authenticating to 
Oracle is to have in the working directory (or in a directory
specified by environment variable \code{CORAL\_AUTH\_PATH})
a file \code{authentication.xml} that looks like the following:

\begin{verbatim}
<connectionlist>

<connection name="oracle://ATLAS_COOLPROD/ATLAS_COOL_PIXEL">
  <parameter name="user" value="ATLAS_COOL_PIXEL_W"/>
  <parameter name="password" value="SomePassword"/>
</connection>

<connection name="oracle://ATONR_COOL/ATLAS_COOLONL_PIXEL">
  <parameter name="user" value="ATLAS_COOLONL_PIXEL_W"/>
  <parameter name="password" value="AnotherPassword"/>
</connection>

</connectionlist>
\end{verbatim}

Then one can, for example, use the string
``\code{oracle://ATLAS\_COOLPROD/ATLAS\_COOL\_PIXEL}'' as the
``connection'' parameter in the \code{CoralDB} constructor.
For working with SQLite, no setup is needed and one can use a connection
string like ``\code{sqlite\_file:myfilename.db}''.


%% ================================================================
\subsubsection{CoralDB public types}

This section briefly explains the meaning of some CoralDB classes.
For more detail and for other classes see the CoralDB header files.

\begin{description}
  
\item[Alias] A triplet of strings that fully defines an alias:
  \code{alias()}, \code{convention()}, \code{id()}.
  
\item[CompoundTag] Represents a set of tags that fully determine the
  database version: strings {\rm \code{}objectDictionaryTag()},
  \code{connectivityTag()}, \code{dataTag()}, \code{aliasTag()}.  Note
  that a \code{CompoundTag} does not have to be fully initialized.
  The \code{isValid()} method tells if a given compound tag object can
  possibly specify a valid combination of tags for use by CoralDB.  A
  completely empty object is valid (some CoralDB operations do not use
  tags), as well as an object where only \code{objectDictionaryTag}
  (IDTAG), or IDTAG and some or all of other tags, defined.  The case
  when any tag is defined but the IDTAG is not does not make sense, so
  for such combinations isValid() returns false.  The isValid() method
  does not check the database to see if the specified tags actually
  exist.

\item[TagStatus] Tag name and whether it is locked. Also used for IDTAGs.

\item[IdTagList] A container of \code{TagStatus}. (Currently vector.)

\item[TagList] A container of \code{TagStatus}. (Currently vector.)

\item[Connection] A complete representation of a CoralDB connection:
  strings \code{fromId()}, \code{fromSlot()}, \code{toId()},
  \code{toSlot()}.
%  \marginpar{Remove isNull(). BTW, the
%  logic in Connection.cpp seems to be inverted.}

\item[TagLocked] Derived from \code{std::runtime\_error}. This is
  thrown on an attempt to modify a locked tag.
  
\item[IdTagUnlocked] Derived from \code{std::runtime\_error}. This is
  thrown on an attempt to lock a tag, for which the corresponding
  IDTAG is not locked.
  
\item[RowNotFound] Derived from \code{std::runtime\_error}.
%
%  \marginpar{Change behaviour of findConnection() to throw an
%  exception?}
%
  This is a generic error that can be thrown by several
  methods when a query fails to return requested data; for example
  because of a mistake in a tag specification.

\end{description}

%% ================================================================
\subsubsection{CoralDB public methods}
%\subsubsection*{CoralDB public methods}

%\vspace{3ex}
%\centerline{\textbf{CoralDB public methods}}
%\vspace{1ex}


\begin{description}
\item[CoralDB]\args{(const string\& connection,\\
  coral::AccessMode accessMode,\\
  seal::Msg::Level outputLevel,\\
  bool autoCommit)} The constructor. 

  The \args{connection} string specifies the database to connect to,
  e.g.\\ \code{oracle://ATLAS\_COOLPROD/ATLAS\_COOL\_PIXEL}\\ 
  or \code{sqlite\_file:test.db}.
  
  The \args{accessMode} is \code{coral::ReadOnly} or
  \code{coral::Update}.

  The \args{outputLevel} is passed to the SEAL message service to set
  its verbosity level.  The verbosity level can be changed later, see
  \code{setOutputLevel()}.

  The \args{autoCommit} can be set to \code{AUTOCOMMIT} or
  \code{MANUAL\_COMMIT} and determines CoralDB's mode of operation.
  In the \code{AUTOCOMMIT} mode, each CoralDB method commits its
  modifications to the database.  In the \code{MANUAL\_COMMIT} mode,
  one must explicitly initiate a write transaction with the
  \code{transactionStart()} call.  (Read transaction in CoralDB are
  transparent to the user in any mode.) Then an arbitrary number of
  changes to the database can be made, which will become permanent
%
%  \marginpar{Is our Oracle DB set up to (an equivalent of) READ
%  COMMITTED?}
%
  and visible to other DB clients only after they are all
  committed at once using \code{transactionCommit()}.  One can get a
  significantly higher performance in the \code{MANUAL\_COMMIT} mode
  (\ref{performance}).  Also, in the \code{MANUAL\_COMMIT} mode one
  can use \code{transactionRollBack()} instead of
  \code{transactionCommit()} to discard all changes since the last
  commit.


\item[void setMsgLevel]\args{(seal::Msg::Level outputLevel)} Changes
  the SEAL verbosity level.

\item[void transactionStart()] Starts a new write transaction in the
  manual commit mode.  Transaction should not be nested, but each
  should be closed with either \code{transactionCommit()} or
  \code{transactionRollBack()}.

  No-op in the autocommit mode.
  
\item[void transactionCommit()] Attempts to commit a
  transaction. No-op in the autocommit mode.

\item[void transactionRollBack()] Discards all DB modifications made
  since the last call to \code{transactionStart()}. No-op in the
  autocommit mode.

\item[bool transactionIsActive() const] True iff a write transaction
  open with \code{transactionStart()} has not yet been committed or
  rolled back.

\item[bool isAutoCommitMode()] Returns the value of autoCommit given
  to the constructor.


%   // Explicit control of caching.

\item[bool getCacheUse() const] True iff the use of cache is enabled.
  Only the lock status of the current tags is being cached.
  \textbf{Note that stale cache data may allow modification of a tag
    that was locked by another client.}

\item[void setCacheUse]\args{(bool state)} Enables (state=true) or
  disables (state=false) cache use. Only the lock status of the
  current tags is being cached.  \textbf{Note that stale cache data
    may allow modification of a tag that was locked by another
    client.}

\item[void updateCache() const] Forces to update the cached values
   by re-reading the database.  

%   // Methods related to schema creation

\item[void createSchema()] Creates a CoralDB schema in the current
  database. Throws an error if the schema already exists.

\item[void dropSchema()] Deletes the schema.  \emph{This implies
  deletion of all CoralDB data in the current database; of all tags,
  including the locked ones.}


%   // Methods dealing with tags

\item[CompoundTag compoundTag() const] returns the current set of tags
  for this CoralDB object.

\item[void setCompoundTag]\args{(const CompoundTag\& existingTag)}  Makes the
  given set of tags current.

\item[void setObjectDictionaryTag]\args{(const string\& existingTag)}
  Sets the current object dictionary tag (IDTAG).

\item[void setConnectivityTag]\args{(const string\& existingTag)}
  Sets the current connectivity tag.

\item[void setDataTag]\args{(const string\& existingTag)}
  Sets the current data tag.

\item[void setAliasTag]\args{(const string\& existingTag)}
  Sets the current alias tag.


%   // Return sorted lists.  The default idTag=="" means use the current idTag.

\item[void getExistingObjectDictionaryTags]\args{(IdTagList\&
  output)}~\textbf{const} Writes the name and the lock status of all
  existing object dictionary tags to its argument.

\item[void getExistingConnectivityTags]\args{(TagList\& output, const
  string\& idTag="")}~\textbf{const} 
  Writes the name and the lock status of connectivity tags defined for
  the current (default) or the given IDTAG.

\item[void getExistingDataTags]\args{(TagList\& output, const string\&
  idTag="")}~\textbf{const}
  Writes the name and the lock status of data tags defined for
  the current (default) or the given IDTAG.
  

\item[void getExistingAliasTags]\args{(TagList\& output, const
  string\& idTag="")}~\textbf{const}
  Writes the name and the lock status of alias tags defined for
  the current (default) or the given IDTAG.
   
%   // Define a new empty tag.

\item[void makeNewObjectDictionaryTag]\args{(const string\& newtag)}
  Creates a new, empty, object dictionary tag.

\item[void makeNewConnectivityTag]\args{(const string\& newtag, const string\& idTag="")}
  Creates a new, empty, connectivity tag for the current (default), or
  a specified object dictionary tag.

\item[void makeNewDataTag]\args{(const string\& newtag, const string\& idTag="")}
  Creates a new, empty, data tag for the current (default), or
  a specified object dictionary tag.


\item[void makeNewAliasTag]\args{(const string\& newtag, const string\& idTag="")}
  Creates a new, empty, alias tag for the current (default), or
  a specified object dictionary tag.

\item[void makeNewCompoundTag]\args{(const CompoundTag\& newtags)}
  A shorthand for the above 4 calls.


%   // These methods define a new tag and copy the content of the current tag into it.
%   // The new tag is associated to the objectDictionaryTag newIdTag, which must already be defined.
%   // If newIdTag is empty (the default), current objectDictionaryTag is used.

\item[void copyConnectivityTag]\args{(const string\& newtag, const string\& newIdTag="")}
Creates a new connectivity tag for the current (default) or a
specified objectDictionaryTag, and copies the content of the current
connectivity tag into it.  

\item[void copyDataTag]\args{(const string\& newtag, const string\& newIdTag="")}
Creates a new data tag for the current (default) or a
specified objectDictionaryTag, and copies the content of the current
data tag into it.  

\item[void copyAliasTag]\args{(const string\& newtag, const string\& newIdTag='''')}
Creates a new alias tag for the current (default) or a
specified objectDictionaryTag, and copies the content of the current
tag alias into it.  

%   // Copies just the object distionary.  The newly created newIdTag will not have any associated 
%   // connectivity/clobs/aliases
\item[void copyObjectDictionaryTag]\args{(const string\& newIdTag)}
This method copies just the object dictionary. The newly created
object dictionary tag will not have any associated tags.


\item[void copyMissingObjects]\args{(const string\& newIdTag, const
  CompoundTag\& from)}

Copies object dictionary entries for objects mentioned in any of the
``from'' tags to newIdTag, skipping those (ID,TYPE) pairs that already
exist in newIdTag.  If only idTag is defined in the ``from'' tags,
adds to newIdTag all pairs that are defined in ``from's'' idTag.

This method can be called before copyConnectivityTag(),
copyAliasTag(), copyDataTag() if the intention is to copy information
to a different IDTAG that may not contain all the necessary objects.

%   // Copy all connectivity, data, and alias tags related to the current objectDictionaryTag,
%   // keeping their names, to the new objectDictionaryTag.
\item[void copyAllForObjectDictionaryTag]\args{(const string\& newIdTag)}
Creates a new object dictionary tag, and copies the full content of
the current object dictionary tag (all associated connectivity, data,
alias tags) into the new one.

%   // Delete all data for the given tag.
%   // If newIdTag is empty (the default), current objectDictionaryTag is used.
\item[bool deleteConnectivityTag]\args{(const string\& tag, const string\& idTag="")}
Deletes the specified connectivity tag for the current (default) or a
specified object dictionary tag.

\item[bool deleteDataTag]\args{(const string\& tag, const string\& idTag="")}
Deletes the specified data tag for the current (default) or a
specified object dictionary tag.

\item[bool deleteAliasTag]\args{(const string\& tag, const string\& idTag="")}
Deletes the specified alias tag for the current (default) or a
specified object dictionary tag.

%// this deletes all related connetivity,data,and alias tags.
\item[bool deleteAllForObjectDictionaryTag]\args{(const string\& idTag)}
Deletes everything associated with the specified object dictionary
tag, and the tag itself.


%// Tag locking
\item[void lockObjectDictionaryTag]\args{(const string\& idTag)} This
  locks just the object dictionary tag, but not the related tags.
  Locking a tag that is already locked has no effect.

\item[void lockConnectivityTag]\args{(const string\& tag, const string\& idTag="")}
  Locks the specified connectivity tag for the current (default) or
  a specified object dictionary tag.
  Locking a tag that is already locked has no effect.

\item[void lockDataTag]\args{(const string\& tag, const string\& idTag="")}
  Locks the specified data tag for the current (default) or
  a specified object dictionary tag.
  Locking a tag that is already locked has no effect.

\item[void lockAliasTag]\args{(const string\& tag, const string\& idTag="")}
  Locks the specified alias tag for the current (default) or
  a specified object dictionary tag.
  Locking a tag that is already locked has no effect.

\item[void lockCompoundTag]\args{(const CompoundTag\& tags)} A
  shorthand for the above 4 lock methods.

\item[void lockAllForObjectDictionaryTag]\args{(const string\& idTag)} 
  Locks the specified object dictionary tag and all related
  connectivity, data, and alias tags.

%// These methods ignore the cached values and read the database on each call.
\item[bool objectDictionaryTagLocked]\args{(const string\& idTag)}~\textbf{const}
  Returns true iff the object dictionary tag is locked.  This method
  does not use cache and reads the database on each call.

\item[bool connectivityTagLocked]\args{(const string\& tag, const string\& idTag="")}~\textbf{const}
  Returns true iff the connectivity tag for the current (default) or a
  specified object dictionary tag is locked.  This method
  does not use cache and reads the database on each call.

\item[bool dataTagLocked]\args{(const string\& tag, const string\& idTag="")}~\textbf{const}
  Returns true iff the data tag for the current (default) or a
  specified object dictionary tag is locked.  This method
  does not use cache and reads the database on each call.

\item[bool aliasTagLocked]\args{(const string\& tag, const string\& idTag="")}~\textbf{const}
  Returns true iff the alias tag for the current (default) or a
  specified object dictionary tag is locked.  This method
  does not use cache and reads the database on each call.

%// Methods for the history table. Default time is current time on the client system (not on the server).

\item[void setHistoricTag]\args{(const CompoundTag\& tag, time\_t  startIOV = 0)}
Makes an entry into the history table.  The default \args{startIOV}
means take the current (on the client system) time.

\item[CompoundTag getHistoricTag]\args{(time\_t when = 0 )}~\textbf{const}
Return the set of tag that was current for the specified time.  The
default is current time on the client system.

\item[void getHistoryTable]\args{(HistoryTable\& result)}
Writes the complete content of the history table into the argument.

\item[bool deleteHistoryEntry]\args{(time\_t exactTimeStamp)} deletes
  the specified history entry.


%// Methods dealing with the master table

\item[void insertMaster]\args{(const string\& id)} Adds the given id
  to the master list table.

\item[bool deleteMaster]\args{(const string\& id)} Deletes the given
  id from the master list table.

\item[vector$<$string$>$ masterList]\args{()}~\textbf{const} Returns the master list.

%// Methods dealing with the object dictionary
\item[bool checkAndInsertObject]\args{(const string\& id, const
  string\& type)} 
  Adds a new object to the object dictionary table, if there is no
  entry there for the \args{id}.
  If there is already an entry for the object, checks that \args{type}
  is the same as already recorded. Throws an exception if the \args{type}
  is different.

\item[unsigned  insertBulkObjects]\args{(istream\& data, bool debug=false)} // returns the number of objects inserted

\item[bool deleteObject]\args{(const string\& id)}
\item[bool updateObject]\args{(const string\& id, const string\& type)}
\item[string objectType]\args{(const string\& id)}~\textbf{const}


\item[bool renameID]\args{(const string\& newId, const string\& oldId)}
Changes oldId to newId (for the current IDTAG) in all tables.  Returns
true on successful update, false if oldId does not exists.

\item[long renameType]\args{(const string\& newType, const string\& oldType)}
Changes oldType to newType in the object dictionary for the current
IDTAG.  Returns the number of updated objects in the dictionary.


\item[void getObjectDictionary]\args{(ObjectDictionaryMap\& objDict,\\
  const CompoundTag\& tags)}~\textbf{const}

  If only object dictionary tag is defined in tags, get the complete
  dictionary for this tag.  If any other tag is also set, get the
  union of sets of IDs mentioned in connectivity, data and alias tags.
  (Not all three types of tags have to be set.)
  
%enum BulkConnectionsFileFormat { SIX\_COL, SIX\_COL\_REVERSED, FOUR\_COL}}
\item[unsigned insertBulkConnections]\args{(istream\& data, BulkConnectionsFileFormat f, bool debug = false)}


\item[void insertConnection]\args{(const string\& id, const string\& slot, const string\& toId, const string\& toSlot)}
\item[bool deleteConnection]\args{(const string\& id, const string\& slot, bool outConnection = true)}
\item[bool updateConnection]\args{(const string\& id, const string\& slot, const string\& toId, const string\& toSlot, bool outConnection = true)}
\item[vector$<$Connection$>$ findAllConnections]\args{(const string\& id, bool getOutConnections = true)}~\textbf{const}


\item[vector$<$Connection$>$ findConnections]\args{(const string\& id,
  const string\& slotPattern, bool getOutConnections = true)}~\textbf{const}  \\
In CoralDB patterns the only special character is '\%'. It matches any
string of zero or more characters (just like SQL '\%'). The escape
character is '$|$'.


\item[vector$<$Connection$>$ findConnectionsByType]\args{(const string\&
id, const string\& typePattern, bool getOutConnections=true)}~\textbf{const}

\item[vector$<$Connection$>$ findConnectionsFromSlot]\args{(const string\&
slotPattern, const string\& toId)}~\textbf{const}

\item[vector$<$Connection$>$ findConnectionsToSlot]\args{(const string\&
id, const string\& toSlotPattern)}~\textbf{const}


\item[void getConnectionTable]\args{(ConnectionTableMap\& connTable)}~\textbf{const}

// Methods dealing with aliases
// Sloppy means skip unknown IDs read from the stream instead of aborting.
\item[unsigned insertBulkAliases]\args{(istream\& data, bool genericFile = true, bool debug = false, bool sloppy = false)}
\item[void insertAlias]\args{(const Alias\& a)}
\item[void insertAlias]\args{(const string\& alias, const string\& convention, const string\& id)}
\item[bool deleteAlias]\args{(const Alias\& a)} // over-constarined
\item[bool deleteAlias]\args{(const string\& id, const string\& convention)}
\item[vector$<$Alias$>$ findAliases]\args{(const string\& id)}~\textbf{const}
\item[string findAlias]\args{(const string\& id, const string\& convention)}~\textbf{const}
\item[string findId]\args{(const string\& alias, const string\& convention)}~\textbf{const}
\item[void getAliasesTable]\args{(AliasesTable\& aliases)}~\textbf{const} // returns a sorted table


// Methods dealing with configuration data
\item[void insertCLOB]\args{(const string\& id, const string\& field, const string\& clob)}
\item[bool deleteCLOB]\args{(const string\& id, const string\& field)}
\item[bool updateCLOB]\args{(const string\& id, const string\& field, const string\& clob)}
\item[string findCLOB]\args{(const string\& id, const string\& field)}~\textbf{const}
\item[void getClobNames]\args{(ClobNameContainer\& names)}~\textbf{const}
   
\end{description}

\clearpage
\subsection{Perl API}

% Main author: Juerg


For programming web interfaces a scripting language such as Perl is much more convenient
to use than C++. Primarily for this purpose a Perl API to CoralDB has been developed. The current
implementation of the Perl API uses the Perl DBI module to directly access the Oracle (or a SQLite)
database. If necessary, the Perl API could easily be reimplemented on top of the C++ API at a later time.

The Perl API is implemented in package CoralDB.pm. Inline documentation is available (use `perldoc CoralDB.pm`).
Table \ref{PERLAPI} gives a summary of the available functionality.

\begin{table}[h]
\begin{center}
\begin{tabular}{|l|p{11cm}|}
\hline
Method & Description \\ \hline
\hline
setTag & Define the different tags to be used \\ \hline
getAllTags & Get all tags of a given type (e.g. 'connTag')\\ \hline
isLockedTag & Determine if a given tag is locked or not\\ \hline
getTagHistory & Get history of ``current'' tags\\ \hline
masterList & Get list of ``master'' objects \\ \hline
getAllObjectTypes & Get list of all object types \\ \hline
objectType & Get type of a given object\\ \hline
findObjects & Find objects by name (ID), type, or alias\\ \hline
findAllConnections & Get list of all incoming or outgoing connections\\ \hline
findConnections & Get connection originiating from or terminating at a given slot of a given object\\ \hline
findConnectionsByType & Find connections from or to objects of a given type\\ \hline
findConnectionsFromSlot & Find connections from a given outgoing slot\\ \hline
findConnectionsToSlot & Find connections arriving at a given slot\\ \hline
findSoftLinks & Find all incoming or outgoing soft-link connections\\ \hline
getAllAliasConventions & Get list of different conventions for which aliases exist\\ \hline
findAliases & Find all aliases for a given object\\ \hline
findAlias & Find alias in a given convention\\ \hline
findClobs & Get all CLOBS stored for a given object\\ \hline
\hline
dump & Utility function: print internal data memebers of CoralDB (Perl) object\\ \hline
\end{tabular}
\caption[Perl API summary]
{Summary of methods currently available in the Perl API.}
\label{PERLAPI}
\end{center}
\end{table}


\clearpage
\subsection{PixDbInterface}
\label{PixDbInterface}

% Main author: Andrei

In PixDbInterface, a CoralDB object corresponds to a \code{DbRecord} and
the object type is class name.  Connections are DbRecordIterators and fields are
CLOBs.  Connectivity is cached on initialization, CLOBs on
access.  A transaction roll back invalidates the caches, and the
\code{PixDbCoralDB} object must be recreated to have its cache in
sync with the database.  
An exception thrown by CoralDB or the underlying libraries may mean
that a transaction has to be rolled back.
The \code{PixDbInterfaceFactory} package
helps to recreate \code{PixDbInteface} objects with consistent
settings.

\clearpage
\subsection{Command Line Tools}

% Main author: Andrei

The CoralDB package includes two C++ programs, \code{ccdb}, and
\code{ccdb-migrate}, and several shell scripts.  The \code{ccdb} program provides a command
line interface to all CoralDB methods.  \code{ccdb-migrate} allows to
dump/restore the content of a CoralDB database to/from a set of
files, and can be used to migrate to a different database schema.


\subsubsection{ccdb}

The ccdb program id documented throught its online help message,
which is printed when the \code{--help} argument is given on the
command line.  The online help message is shown below.

%%----------------------------------------------------------------

\VerbatimInput[fontsize=\small,obeytabs=true]{help.ccdb}

%%----------------------------------------------------------------


\clearpage
\subsubsection{ccdb-migrate}

The \code{ccdb-migrate} program can dump the content of a CoralDB
database into a set of files, and load the content of such a dump
into a database.  The \code{--noSchema} options (see below) allow
to add data to an existing database, so that information from
different databases can be merged.  The \code{--only*Tag} options
allow to dump only selected tags.

The \code{ccdb-migrate} program is linked against the CoralDB
library, and can work only with the schema that is implemented by
that library.  For schema evolution, the following approach can
be used.  First, make a dump of the database to be migrated using
\code{ccdb-migrate} from a CoralDB package version compatible
with the schema of that database.  Then restore the dump using a
version of \code{ccdb-migrate} corresponding to the desired new
schema.   For this to work, \code{ccdb-migrate} has to understand 
the format of dumps made for at least one previous DB schema.

The dump structure consists of a top level directory (whose name
is used on \code{ccdb-migrate} command line), which contains a
\code{CORALDB\_DUMP\_VERSION} file that identifies the version of
the dump format (a single integer that file must begin with), and
other files and subdirectories, which contain DB data.  The files
are tab-separated text files that can be edited by
experts. (Files typically start with a comment line explaining
the content of the file.)

Tabs and newlines in CLOBS are encoded as \verb|\t| and
\verb|\n|, and backslashes as \verb|\\| for storage in the
tab-separated files.  Other than that escaping, CLOBs are dumped
literally.

Entries in the files are sorted in a reproducable order to
facilitate comparison of different DB versions using
\code{diff~-r}.

The \code{ccdb-migrate} online help message is shown below:

%----------------------------------------------------------------

\VerbatimInput[fontsize=\small,obeytabs=true]{help.ccdb-migrate}

%----------------------------------------------------------------


\clearpage
\subsubsection{ccdb-makedb}

\code{ccdb-makedb} is a simple shell script to help creating a connectivity database from scratch using
\code{ccdb}. It takes as input a tab-separated-value files of connections and of aliases, optionally creates the CoralDB database schema, and imports the input files into the database. It's usage information is shown below.

%----------------------------------------------------------------
\begingroup
\small

\begin{verbatim}
Usage: ccdb-makedb [-c] [-d] dbconn idtag tag [connectivityfile aliasesfile]
\end{verbatim}

\endgroup
%----------------------------------------------------------------



\clearpage
\subsection{CoralDB Browser}

% Main author: Juerg

The CoralDB browser is a web application that allows interactive browsing of a
CoralDB connectivity database. It has the following features:
\begin{itemize}
\item Provide an overview of the database contents including available tags and history of what tag was current at a given time
\item Search for objects by their ID, type or aliases
\item Display names using IDs or any other alias convention
\item Display connections to and from an object using hyperlinks to allow easy navigation within the connectivity graph
\item Display or download connectivity graph for a particular object
\item Display payload data
\item Display master list
\item Generate tables of properties from selected events (see section \ref{CORALDBTABLES})
\end{itemize}

Figures \ref{CORALDBBROWSER-overview} and \ref{CORALDBBROWSER-example} show screen dumps from the CoralDB browser. CoralDB browsers
for the pixel connectivity database can be accessed at \cite{CORALDBBROWSER-url} (using the master database on the
ATONR\_COOL database server) or at \cite{CORALDBBROWSER-replica-url} (using the offline replica on the ATLAS\_COOLPROD database server). Note that the first few lines of the overview page show the database connection and table prefix (normally
{\tt ATLAS\_COOLONL\_PIXEL.CORALDB2} being used).

\NFIG{t}
{coraldboverview.eps}{width=\textwidth}
{CORALDBBROWSER-overview}{CoralDB browser: overview page}
{The CoralDB browser overview page gives an overview of the contents of the CoralDB database. It shows which tags of
connectivity, aliases and payload are available for each object dictionary version (ID tag), and which of those are locked.
In addition, it provides convenient links to tools for generating tables and to documentation.}

\NFIG{t}
{coraldbbrowser.eps}{width=\textwidth}
{CORALDBBROWSER-example}{CoralDB browser: example page for a PP0 object}
{The CoralDB browser allows searching of objects by name, alias, or type (wildcards are allowed). Once an object is selected,
it is easy to navigate to a connected object by clicking on the corresponding link.Connectivity graphs can
optionally be displayed within the browser page, or can be downloaded as a PostScript file. Aliases and payload data are also displayed.}



\clearpage
\subsection{CoralDB Tables}
\label{CORALDBTABLES}

% Main author: Juerg

It is useful to be able to generate spreadsheet-like tables of how
objects are connected with each other directly from the connectivity
database. A set of ``standard'' tables can be selected from the Connectivity Tables web page that is
linked to the overview page of the CoralDB browser. Customizable tables can be generated
either for objects selected in the CoralDB browser, or, more generally, for a user-specified list of
objects through the Table Maker web page (also linked from the overview page). An example of a table
is shown in figure \ref{CORALDBTABLES-fig}.

\NFIG{t}
{coraldbtable.eps}{width=\textwidth}
{CORALDBTABLES-fig}{CoralDB Table}
{The CoralDB browser allows the generation of tables for a selected set of objects. In this example, all
PP0 objects were selected. Users can use a predefined table as a starting point (here the PP0-LV table was chosen) and
then adapt the table to their needs, or can start from scratch with a fresh table. Error messages are shown if a rule
cannot be executed for a particular cell (see bottom-right of the figure for an example), making it easy to spot possible mistakes in the connectivity database.}


In order to extract from the connectivity database the information
needed in the table for each object, one starts at the node in the
connectivity graph corresponding to a particular object and follows a certain
prescription (or rule) for traversing the connectivity graph until one
arrives at the desired information.

For example, in order to extract the name of the optoboard to which a
given pixel module is connected, one can start at the node
corresponding to that module and use the following prescription:

\begin{enumerate}
\item Go up to the parent node using the slot named UP
\item Go up again to its parent node using the slot named OB
\item Get the ID of the resulting node
\end{enumerate}

The syntax for writing such rules is summarized in table \ref{CORALDBRULES} and
a few examples are given in table \ref{CORALDBRULES-examples}.
Thus the above prescription would be written as
\begin{verbatim}
U=UP U=OB ID
\end{verbatim}

As can be seen in figure \ref{CORALDBTABLES-fig}, the web form allows to either specify
the rule to be used for each column of the table explicitly, or one
can choose from a set of predefined rules or complete tables (provided
that the table is generated for objects of a single type).

Predefined rules and tables are stored in configuration files in
package {\tt CoralDB} in subdirectory rules with file names of the form
{\tt TYPE.rules} (definition of rules for objects of type TYPE) and
{\tt TYPE-TABLENAME.table} (definition of a table TABLENAME for objects of
type TYPE). The first column in the file is the name of the rule or
the column title. The second column is separated by a single tab
character and contains the rule or column definition. To define
the column contents, both the names of rules defined for that type of
object and individual commands as described above may be used.


\begin{table}[tbp]
\begin{center}
\begin{tabular}{|l|l|}
\hline
Command & Description \\ \hline
\hline
\multicolumn{2}{|l|}{\bf Commands for writing out information:} \\ \hline
ID & 	Write ID of current node \\ \hline
TYPE & 	Write type of current node \\ \hline
SLOT & 	Write arrival SLOT on current node\\ \hline
FROMID & 	Write ID of previous node \\ \hline
FROMSLOT & 	Write departure SLOT on previous node\\ \hline
NU & Number of upward connections \\ \hline
NU!n, & Error message if number of upward connections is not equal to, \\
NU$>$n, NU$<$n & greater than, less than n \\ \hline
ND & Number of downward connections \\ \hline
ND!n, & Error message if number of downward connections is not equal to, \\
ND$>$n, ND$<$n &  greater than, less than n \\ \hline
"word" & 	Write ``word'' (cannot contain any white space) \\ \hline
A=convention  &  Write alias for current node using ``convention'' \\
A!convention  &  \\ \hline
\multicolumn{2}{|l|}{\bf Commands for navigation:} \\ \hline
U=slot, U!slot  & Go up using slot \\ \hline
UF=slot, UF!slot & Go up using connection from slot \\ \hline
D=slot, D!slot  & Go down using slot \\ \hline
DE=slot, DE!slot & Go down using connection ending at slot \\ \hline
T=type, T!type & Go to node(s) of given type (both up and down nodes are searched) \\ \hline
TU=type, TU!type & Go to node(s) of given type (only up nodes are searched) \\ \hline
TD=type, TD!type & Go to node(s) of given type (only down nodes are searched) \\ \hline
\multicolumn{2}{|l|}{\bf Commands for formatting:} \\ \hline
HTMLBOLD & Write subsequent text in bold font \\ \hline
HTMLITALIC & Write subsequent text in italic font \\ \hline
HTMLRED & Write subsequent text in red \\ \hline
\end{tabular}
\caption[Rules for generating tables]
{Summary of rules for generating tables from a CoralDB connectivity
database. Each rule consists of a series of commands separated by
white space and is used to either navigate from the current
node to another node, or to write out some information.
The processing of a rule continues until either one arrives at
the end of the rule, or until one of the commands cannot be executed,
for example because a specified connection doesn't exist. In the
latter case, depending on the form of the command used, the rule
terminates silently (rules using '='), or outputs an error message
(rules using '!'). Some commands may lead to several nodes. In this case, rule processing
continues for each of the nodes, and each of the selected paths can
result in some output.
The directions "up" and "down" refer to travelling along incoming and outgoing connections, respectively.
Wildcards ('\%')are allowed in the specification of both slots and object type.}
\label{CORALDBRULES}
\end{center}
\end{table}


\begin{table}[p]
\begin{center}
\begin{tabular}{|p{5.5cm}|p{5.5cm}|p{4.5cm}|}
\hline
Rule & 	Sample output & 	Description \\ \hline \hline
ID & 	D1A\_B01\_S1\_M1 & 	ID of current object \\ \hline
U=UP A=GEOID & 	SQP-A12-OP-C2-P4L-T & 	Name of parent node using GEOID alias \\ \hline
U=UP A=GEOI & 	 & No output, because A= is used and there is no alias GEOI \\ \hline
U=UP A!GEOI & ERROR: No alias for convention GEOI for D1A\_B01\_S1 & Error, because A! is used and there is no alias GEOI \\ \hline
U=UP D=\% FROMSLOT & M1 \newline
M2 \newline
M3 \newline
M4 \newline
M5 \newline
M6 & 	List of outgoing slots on previous node \\ \hline
T=PP0 T=\% TYPE & 	CONTAINER \newline
OPTOBOARD \newline
HV-PP1 \newline
LV-PP1 \newline
OPTO-PP1 \newline
MODULE \newline
MODULE \newline
MODULE \newline
MODULE \newline
MODULE \newline
MODULE & 	List of the type of all objects connected to a PP0 (which is the parent of a module; you could equally use U=UP instead of T=PP0) \\ \hline
U=UP U=OB T=RODBOC ID T=RODCRATE ID & 	ROD\_C1\_S11 ROD\_CRATE\_1 & 	Get ROD slot and crate (note that you can output several pieces of information in the same rule) \\ \hline
\end{tabular}
\caption[Examples of rules]
{Examples of rules for generating tables from a CoralDB connectivity
database. The sample output is given for applying the rules to module D1A\_B01\_S1\_M1.}
\label{CORALDBRULES-examples}
\end{center}
\end{table}

\endgroup
