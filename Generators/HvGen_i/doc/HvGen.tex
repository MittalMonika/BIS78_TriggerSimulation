\documentclass[11pt]{article}
\newdimen\SaveHeight \SaveHeight=\textheight
\textwidth=6.5in
\textheight=8.9in
\textwidth=6.5in
\textheight=9.0in
\hoffset=-.5in
\voffset=-1in
\def\topfraction{1.}
\def\textfraction{0.}   
\def\topfraction{1.}
\def\textfraction{0.}           
\begin{document}
\title{HvGen\_i: An interface between HiddenValley Generator and Athena\\
Version in release 13.0.0 and later}
\author{ Stefano Giagu (Stefano.Giagu@cern.ch) }
%\today

\maketitle           

This package runs HV generator from within Athena. \\See the example
in {\bf HvGen\_i/share/jobOptions.HvGenPythia.py } which show how to
read HV events and hadronize them using Pythia.

Users must first run HV Generator in standalone mode and make a file of events. An athena job
then takes these events hadronizes them and passes them down the
Athena event chain. The events must be made with a version of HV generator
that is compatible.
Compatible versions can be found in:
/afs/cern.ch/user/g/giagu/public/HvGen and /afs/cern.ch/atlas/offline/external/hvgen

To hadronize HV generated events with Pythia, you only need to run athena with the jobOptions
file jobOptions.HvGenPythia.py by typing in the prompt \\
{\it athena jobOptions.HvGenPythia.py}\\

More infomation about Hidden Valley Models and HV generator here: \\
/afs/cern.ch/user/g/giagu/public/HvGen/doc \\
/afs/cern.ch/atlas/offline/external/hvgen/doc and in: \\
http://www.phys.washington.edu/users/strasslr/hv/hv.htm

\end{document}
