\documentclass[11pt]{article}
\newdimen\SaveHeight \SaveHeight=\textheight
\textwidth=6.5in
\textheight=8.9in
\textwidth=6.5in
\textheight=9.0in
\hoffset=-.5in
\voffset=-1in
\def\topfraction{1.}
\def\textfraction{0.}   
\def\topfraction{1.}
\def\textfraction{0.}           
\begin{document}
\title{Acermc\_i: An interface between Acermc and Athena\\
Version in release 6.5.0 and later}
\author{  Borut Kersevan (Borut.Kersevan@cern.ch), Ian Hinchliffe (I\_Hinchliffe@lbl.gov) and Georgios Stavropoulos (George.Stavropoulos@cern.ch) }
%\today

\maketitle           

This package runs Acermc  from within Athena. \\See the example
in {\bf Acermc\_i/share/jobOptions.AcermcPythia.py } and {\bf
  Acermc\_i/share/jobOptions.AcermcHerwig.py }  which show how to
read Acermc events and hadronize them using Pythia or Herwig

Users must first run 
Acermc in standalone mode and make a file of events. An athena job
then takes these events hadronizes them and passes them down the
Athena event chain. The events must be made with a version of Acermc
that is compatible, recent versions that support the Les Houches
interface should be acceptable. There is a compatible version in  /afs/cern.ch/atlas/offline/external/acermc


To hadronize {\bf AcerMC} generated events with Herwig, you only need to run athena with the jobOptions
file jobOptions.AcerMCHerwig.py by typing in the prompt \\
{\it athena jobOptions.AcerMCHerwig.py}\\

To hadronize {\bf AcerMC} generated events with Pythia, you only need to run athena with the jobOptions
file jobOptions.AcerMCPythia.py by typing in the prompt \\
{\it athena jobOptions.AcerMCPythia.py}\\


More infomation about ACerMC  here

http://borut.home.cern.ch/borut/

\end{document}







