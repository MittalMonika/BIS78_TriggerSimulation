\documentstyle[12pt,epsfig]{article}
%\documentclass[11pt]{cernrep}

\topmargin 0cm
%\headheight 0pt
%\headsep 0pt
\topskip 0pt

\oddsidemargin  -5mm
%\evensidemargin -3mm
\textheight 22cm
\textwidth 160mm
%\parindent 7mm
\itemsep -5mm
\newcommand{\tabulka}[3]{
  \begin{table}[h]
    \begin{center}
      #1
      \end{center}
   \caption{#2}
    \label{#3}
 \end{table}}

\newcommand{\psubt}{\mbox{$p_{T}\ $}}
\newcommand{\bquark}{\mbox{$\rm b\ $}}
\newcommand{\abquark}{\mbox{$\rm \overline{b}\ $}}
\newcommand{\bab}{\mbox{$\rm b\overline{b}\ $}}

\begin{document}
\title{\Large{ \bf PythiaB} \\
\large{\bf interface to Pythia6 dedicated to simulation of beauty events.}}
\author{ \normalsize Maria Smizanska (maria.smizanska@cern.ch)\\
  University Lancaster}


\normalsize
%\today

\maketitle


%\begin{tabular*}{\textwidth}{@{}l@{\extracolsep{\fill}}rrr}
\begin{scriptsize}
\begin{tabular}{|l|l|l|l|}
\hline

release 3.1.0  & PythiaBModule - algorithm & new  & M.Smizanska based on  \\
  April 2002   &    and related code       &      & PythiaModule and atgenb \\ \hline
release 4.4.0  & PythiaBModule             & update    &  \\
 Oct.2002      & Event store at repeated hadronization  & &  M.Muller,  P.Calafiura      \\
               & bbbb, bbcc if forced decay& & M.Smizanska \\ \cline{2-4}

               & BSignalFiler - algorithm & new &   M.Muller based on \\
           &                          &  &EventFilter and HistSample \\
\hline
release 5.0.0  & Btune.py & new    &  M.Smizanska \\
 2.Dec.2002              &           &          & Pythia6 B-tuned parameters \\ \cline{2-4}
               &  $\rm PythiaB\_Signal.py$    & updated & M.Smizanska\\
               &  $\rm PythiaB\_bbmu6X.py$     &     "  &  " \\ \hline
 release 6.5.0 & PythiaBModule renamed to PythiaB                        & reorganization   & G.Stavropoulos,  \\
 3.Aug.2003    &   replaced from former & of Generators  &I.Hinchcliffe\\
               & /Generators/GeneratorModules/PythiaBModule & &\\
               & to /Generators/PythiaB  & &\\ \cline{2-4}
               &  $\rm PythiaB\_Signal.py$    & updated to  & M.Smizanska\\
               &   $\rm PythiaB\_bbmu6X.py$ &  new  Rndm Number     &  \\ 
               &    &    service      &   \\ \hline
\end{tabular}

\end{scriptsize}
\tableofcontents
\newpage
\section{Shortly}
\begin{tabular}{|l|l|}
 \hline
& \\
 \bf{athena $\rm PythiaB\_bbmu6X.py$} &simulate $\rm \bab \rightarrow
\mu(6GeV) X$ \\
& \\
\hline & \\
\bf{athena  $\rm PythiaB\_Signal.py$} & simulate one exclusive B-channel \\
   &     default   $ \rm B_{s} \rightarrow J/\psi(\mu\mu) \phi $ \\
   &     other examples  inside $\rm PythiaB\_Signal.py $\\
\hline
\end{tabular}
\section{What PythiaB provides }

{\bf PythiaB} provides an interface to Pythia6 allowing to: {\bf
1.}~speed up B events simulations, {\bf 2.}~simulate only wanted
decay channel,  {\bf 3.}~apply selection cuts organized in several
levels: after parton showering ( before hadronization), after
hadronization: trigger-like cuts and off-line type of cuts, {\bf
4.}~define b-production  parameters - optimal parameters are
prepared as default. User can control 1.-4. by datacards contained
in jobOptions.py file.

  PythiaB inherits from Pythia and GenModule, so the event is converted
  from Pythia common block HEPEVT to c++  HepMc tree structure
  and stored in Storegate as a transient object and finally written to a root persistency.
  
\section{Where to use PythiaB}

PythiaB can be used for a generation of beauty events 
written into a persistent root file, which can serve as an input into 
a detector simulation
(atlsim) or into ATFAST. PythiaB can also work in the same job with ATLFAST 
  in this case the HepMC tree is passed from PythiaB to ATLFAST in
  a transient form. 
\section{How to use PythiaB}



\subsection{Just changing jobOptions.py, linking only PythiaB files}


To work just with PythiaB avoiding to connect large number of
 unwanted files to your
TestRelease area      you simply do following:


\begin{enumerate}

\vspace{-3mm}\item set up athena under cmt,  see cmt manual
\cite{cmt} and Athena manual \cite{athena}.
\vspace{3mm}\item cmt co TestRelease 
\vspace{-3mm} \item cd TestRelease/TestRelease-00-*/cmt
\vspace{-3mm} \item edit requirements
and change following lines:
%\end{enumerate}

\begin{verbatim}

     O R I G I N A L 
  
 use AtlasRelease AtlasRelease-* 
  
\end{verbatim}

\begin{verbatim}

     R E W R I T T E N 
  
#use AtlasRelease AtlasRelease-*
use PythiaB PythiaB-00-* Generators
use AthenaCommon AthenaCommon-* Control
use AtlasROOT           AtlasROOT-*        External  
use EventHdrAthenaRoot EventHdrAthenaRoot-* Event
use EventSelectorAthenaRoot EventSelectorAthenaRoot-* Database/AthenaRoot


\end{verbatim}

%\begin{enumerate}
 \vspace{-3mm} \item  source setup.sh 
\vspace{-3mm}\item  gmake
\vspace{-3mm} \item  cd TestRelease/TestRelease-00-*/run
\vspace{-3mm} \item  edit and adopt jobOptions.py to your needs
 \vspace{-3mm}\item  to run interactively: athena jobOptions.py $>$output.log
\end{enumerate}








\subsection{Changing an existing Athena code}
I want to change some of the existing code, e.g. apply dedicated
selection criteria in user$\_$finsel.F, or change PythiaB.cxx.
Then do following: \\

\begin{enumerate}
\vspace{-3mm}\item set up Athena under cmt
\vspace{-3mm}\item cmt co TestRelease
\vspace{-3mm}\item cmt co Generators/PythiaB
\vspace{-3mm}\item cd Generators/PythiaB/src/
\vspace{-3mm}\item edit and adopt user$\_$finsel.F
\vspace{-3mm}\item edit and adopt PythiaB.cxx
\vspace{-3mm}\item cd TestRelease/TestRelease-00-*/cmt
\vspace{-3mm}\item change the default requirements to the
requirements as written at the item 4 of the  subsection 4.1
\vspace{-3mm}\item source setup.sh
\vspace{-3mm}\item cmt broadcast gmake
\vspace{-3mm}\item cd TestRelease/TestRelease-00-*/run
\vspace{-3mm}\item (edit and adopt jobOptions.py if you need)
\vspace{-3mm}\item to run interactively: athena jobOptions.py $>$
output.log
 \end{enumerate}
 
 The release 6.5.0 allows also running in the InstallArea, 
 so you can as well do the last three points as follows:
 
 \begin{itemize}
 \vspace{-3mm}\item[11.] cd InstallArea/jobOptions/PythiaB 
\vspace{-3mm}\item[12.] (edit and adopt jobOptions.pyif you need)
\vspace{-3mm}\item[13.] to run interactively: athena jobOptions.py $>$
output.log
 \end{itemize}
 

\subsection{Changing an existing Pythia6 subroutine }
I have my own physics model in fortran.
I want to pluck it into an existing PYTHIA6 subroutine PYxyz.F.
It is similar to previous case. 
So you do following: \\



\begin{enumerate}
\vspace{-3mm}\item set up Athena under cmt
\vspace{-3mm}\item cmt co TestRelease
\vspace{-3mm}\item cmt co Generators/PythiaB
\vspace{-3mm}\item cd Generators/PythiaB/src/
\vspace{-3mm}\item cp ${\$}$MYSPACE/PYxyz.F ./
\vspace{-3mm}\item cd Generators/PythiaB/cmt/
edit requirements 
and change following lines:

\begin{verbatim}

     O R I G I N A L 
  
                                       charm.F \
                                       children.F \  
  
\end{verbatim}

\begin{verbatim}

     R E W R I T T E N 
                                       charm.F \
               PYxyz.F  \
                                       children.F \  


\end{verbatim}

\vspace{-3mm}\item cd TestRelease/TestRelease-00-*/cmt
\vspace{-3mm}\item change the default requirements to the
requirements as written at the point 4 of the  subsection 4.1
\vspace{-3mm}\item source setup.sh
\vspace{-3mm}\item cmt broadcast gmake
\vspace{-3mm}\item cd TestRelease/TestRelease-00-*/run
\vspace{-3mm}\item (edit and adopt jobOptions.py if you need)
\vspace{-3mm}\item to run interactively: athena jobOptions.py $>$ output.log




 \end{enumerate}

  or
  
  
\begin{itemize}
 \vspace{-3mm}\item[11.] cd InstallArea/jobOptions/PythiaB 
\vspace{-3mm}\item[12.] (edit and adopt jobOptions.py if you need)
\vspace{-3mm}\item[13.] to run interactively: athena jobOptions.py $>$
output.log
 \end{itemize}
 
\section{Why dedicated code for b-physics }

\subsection{Speed up simulation}
Pythia6 is a phenomenological model that provides  three
mechanisms to produce b-quark. They are classified as a flavour
creation ($ \rm gg \rightarrow b \overline{b}$, $ \rm qq
\rightarrow b \overline{b}$), flavour excitation ($\rm gb
\rightarrow gb$) and gluon splitting ($g \rightarrow b
\overline{b}$). All these three mechanisms are activated if
following Pythia processes are allowed:
 isub=11 f + f' $\rightarrow$ f + f' (QCD) ,
 isub=12 f + fbar $\rightarrow$ f' + fbar' ,
 isub=13 f + fbar $\rightarrow$ g + g  ,
 isub=28 f + g $\rightarrow$ f + g   ,
 isub=53 g + g $\rightarrow$ f + fbar ,
 isub=68 g + g $\rightarrow$ g + g ,
  with parameter "pysubs msel=1".
 In this regime beauty quark is
produced approximately in $1\%$ of events (the number depends on
the selected phase space). There is another mechanism which is
activated by selecting "pysubs msel 5" - here b-quark is produced
in each event in a hard scattering processes of type: $ \rm gg
\rightarrow b \overline{b}$, $ \rm qq \rightarrow b \overline{b}$
using mass matrix elements. This mechanism however does not
describe known b-production data from Fermilab. The difference is
so large that it is not suggested to use it for simulations. The
details can be found for instance in \cite{my5}.

In order to speed up the simulation in "pysubs msel 1" mode
PythiaB interrupts a simulation after the parton development (just
before the hadronization) to check for the presence of $b
\overline{b}$ quarks satisfying user's defined limits in \psubt
and $\eta$. If user wants the hadronization is repeated several
(MHAD) times using the same parton part of the event. The
resulting cross section is then divided by MHAD. It is suggested
that user selects MHAD such that the average number of accepted
events with the same parton part is close to 1. The latter value
is printed out by PythiaB in an output log file.


\subsection{Simulate only wanted channel}
It is not difficult to select only wanted channel in Pythia.
PythiaB does nothing special - just saves your time and work:
1.~by providing an interface to datacards allowing user to close
and open channels without an intervention in the code, 2.~by
providing special datacard files allowing to close large groups of
unwanted channels. For more details see section 6.2. PythiaB
provides also datacards allowing user  to apply trigger-like cuts
and offline-like selection cuts.

\subsection{How should I calculate a cross section}

If a user  did not force any of B-decays, then the  cross section
is calculated in PythiaB and appeares in the output.log file under
the name:
  CROSSSECTION OF YOUR B-CHANNEL IS. If you forced any channel in B-decay chain
  you should multiply the PythaBModule cross section by a branching ratio for this
  channel.
\tabulka{\begin{tabular}{||c|c|c|c||} \hline 
 
 MEANING     &     ATLAS VALUE     &           Pythia6 default \\
 \hline
 b-quark production-related parameters              &                     &\\          
 Structure fuction                     &   "pypars mstp 51 1 (CTEQ3)" &   CTEQ5 \\       
Min bias               &          "pysubs msel 1", &  1 \\
Max parton virtuality           &  &  \\
factor to multiply $Q^2_{\mbox{hard}}$ &  "pypars parp 67 1", &  1\\
The factorization scale  $Q^2_{\mbox{hard}}= $&                     &\\
$\psubt^2 (P_1^2 + P_2^2 + m_3^2+m_4^2)/2$  &          "pypars mstp 32 8"  &           8 \\

\hline     
B hadron related parameters                 &                     &\\


Spin s=1 probability            &  "pydat1 parj 13 0.65"                   &  0.75\\ 
j=1  l=1 s=0                    &        "pydat1 parj 14 0.12", &     0\\
j=0  l=1 s=1            &        "pydat1 parj 15 0.04", &     0\\
j=1  s=1 l=1            &        "pydat1 parj 16 0.12", &     0\\
j=2  s=1 l=1                    &        "pydat1 parj 17 0.2", &     0\\
Peterson fragmentation   $\epsilon_b$      &   "pydat1 parj 55 -.006"                 & -.005   \\    
           
            
No B-oscillations            & "pydat1 mstj 26 0", &\\
            
         
         \hline
            
Multiple   interactions parameters   &                     &\\
Model           & "pypars mstp 82 4 (double gauss)", & 1 (step fcn)\\
 Regularization \psubt scale          & "pypars parp 82 1.4",  &     1.9\\
 Double gauss parameters           &"pypars parp 83 0.5",   &    0.5\\
   "        & "pypars parp 84 0.4",   &    0.2\\
  Gluon probability          &"pypars parp 85 0.9",   &    0.33\\
 Two gluon probability          & "pypars parp 86 0.95",  &    0.66\\
 Energy scale for parp 82          & "pypars parp 89 1800",  &    1000\\
 Power of energy rescaling           &"pypars parp 90 0.25".  &    0.16 \\ 
            
         \hline 
  \end{tabular}}
           {The optimized set of B-physics-related Pythia6 parameters, values  for ATLAS simulations.}
           {table:abab} 







\subsection{Optimal B-simulation parameters}
PythiaB provides an interface to datacards allowing user to define
beauty production  parameters. The optimal  values of these
parameters were selected in \cite{my6} and are summarized in
the  Table \ref{table:abab}. This set is provided by PythiaB
as a default.
  The new structure functions CTEQ5
('mstp 51 7' ) fails to describe  the b-production data
while the older CTEQ3 ('mstp 51 1') fits fairly well \cite{my6}.
The reason is that Pythia contains many parameters that influence the
b-production and a complex tuning have been done 
by phenomenologists only for CTEQ3, \cite{Py6,Py6Nor}. For CTEQ5 the job still needs to be
done.  Such a  job exceeds possibilities of our
group. Currently we recommend user to use 'mstp 51 1' in a combination with the other 
parameter values of the optimized set provided by PythiaB.  
%We do not
%see this inconsistent with the fact that ATLAS minimum bias events
%are simulated with CTEQ5.




\section{Datacards}


In Athena a user controls the simulation using the datacards
contained in files jobOptions.py. For PythiaB  you will find in
release two prepared files: {\bf $\rm PythiaB\_bbmu6X.py$} and
{\bf $\rm PythiaB\_Signal.py$} from which you can easily derive
your case. File {\bf $\rm PythiaB\_bbmu6X.py$} simulates events
$\bab \rightarrow \mu6 X$. File {\bf $\rm PythiaB\_Signal.py$}
simulates one exclusive B-channel ( default is for $ \rm B_{s}
\rightarrow J/\psi(\mu\mu) \phi $).


\subsection{ Job control Datacards  }
\begin{tabular}{|p{75mm}|p{90mm}|}
\hline  & \\

\verb!Generator.Members = [!& \\ \hline
 \verb!"PythiaB",! & Simulate b-event in HepMc into  transient
 store\\ \hline
 \verb!"DumpMC" ,!  & Dump HepMc for each accepted event into  log file\\
             & do not use with large statistics \\ \hline
 \verb! "BSignalFilter"]!    & Finds all B-decay chains and store in ntuple\\
    &  for print  level=2 also dumps B chains into log file \\
 \hline
\verb!ApplicationMgr.EvtMax = 10! & Number of events to be accepted \\
\hline
\verb!EventSelector.RunNumber = 1;!& Defines a Run numb. in the job \\
\hline
\verb!EventSelector.FirstEvent  = 1;!& Defines the first evt numb. in the job\\
\hline
\verb!AtRndmGenSvc.Seeds =! & User random number seeds\\
\verb!["PYTHIA 4789899 989240512", ! &\\
\verb!"PYTHIA_INIT 820021 2347532"]! &\\ \hline


 
\end{tabular}



\subsection{  Opening - closing decay channels }



\begin{tabular}{|p{75mm}|p{90mm}|}
\hline  & Channel\\ \hline
  & \\
\verb!#include "CloseAntibQuark.py";
      PythiaB.PythiaCommand+=[
                                    "pydat3 mdme 1120 1 1",
                                    "pydat3mdme 996 1 0",
                                    "pydat3 mdme 998 1 0" ]! &
    $ \rm B_{s} \rightarrow J/\psi(\mu\mu) \phi $                                 \\

 & \\ \hline
\verb!#include "CloseAntibQuark.py";
      #include"Dsphipi.py";
      PythiaB.PythiaCommand += [
                "pydat3 mdme 1105 1 1"]! &
 $ \rm B_{s} \rightarrow D_{s} (\phi \pi) \pi$     \\ \hline

& \\
 \verb!#include "CloseAntibQuark.py";
 PythiaB.PythiaCommand+= [
                                      "pydat3 mdme 1027 1 1",
                                      "pydat3 mdme 996 1 0",
                                      "pydat3 mdme 998 1 0" ]! &
$ \rm B_{s} \rightarrow J/\psi(\mu\mu) K^0(\pi^{+}\pi^{-}) $                                      \\
                                      \hline

\end{tabular}



Be carefull: there should be just one space between numbers in
'datacards mdme'. If you put two or more - program will not
complain, but ignore your 'datacard'.

\subsection{ How to change Pythia parameters}

 The optimized Pythia6 B-physics-related parameters (Table.\ref{table:abab})
  are defined in ATHENA by the file Btune.py, 
 which is  included in the standard $\rm PythiaB\_Signal.py$ and  
 $\rm PythiaB\_bbmu6X.py$. To change any value you can just copy a 
 corresponding line from  Btune.py into $\rm PythiaB\_Signal.py$ 
 placing it after the line \verb!#include "Btune.py"! and changing the value
 of parameter to a wanted one. It is suggested that the large background samples 
 of general use are always generated using the values defined in Btune.py.

  
  
%\begin{tabular}{|p{75mm}|p{90mm}|}
%\hline

%\verb!PythiaB.PythiaCommand += { ! & \\

%                \verb!"pysubs ckin 3 10.",! & \psubt limit of hard scattering\\
%                \verb!"pysubs ckin 9 -3.5",! & $\eta$ limit of hard scattering  for the\\
%                \verb!"pysubs ckin 10 3.5",! & \hspace{16mm}  product with larger $|\eta$| \\
%                \verb!"pysubs ckin 11 -3.5",! & $\eta$ limit of hard scattering   for the\\
%                \verb!"pysubs ckin 12 3.5",! & \hspace{16mm} product with smaller $|\eta$| \\
%                \verb!"pysubs msel 1",!  & 1=all \bab production mechanisms  \\
%                                         & 5=\bab massive matrix el. \\
%                \verb!"pydat1 mstj 26 0",! & 0= B mixing off, 1=B mixing on\\
%                \verb! "pypars mstp 51 1"};! & 1=CTEQ3(use CTEQ3) \hspace{3mm}
%                7=CTEQ5\\ \hline


%\end{tabular}



\subsection{ Define your selection cuts   }

\begin{tabular}{|p{85mm}|p{80mm}|}
\hline

\verb!PythiaB.cutbq = !
& b-quark cuts: pT,  eta, and(or)    \\
\verb!["7. 3.5 or 7. 3.5"]! &antib-quark cuts: pT,  eta\\

\hline
\verb!PythiaB.lvl1cut = { 1., 6., 2.5};! & lvl1 single muon cuts: switch on(1)/off(0) pT eta\\

\hline
 \verb!PythiaB.lvl2cut = !&lvl2 second lepton cuts: \\
\verb![ 0., 13., 6., 2.5]!& switch on(1)/off(0),
PID(13mu~or~11e),pT, eta\\ \hline
 \verb!PythiaB.offcut = !  & offline cuts: applied on stable particles \\
 \verb![ 1., 0.5, 2.5, 0.5, 2.5, 0.5, 2.5]!        & at the end of B-decay chain \\
            & switch
on(1)/off(0),  pT, eta of K/pi/p,
                                 pT, eta of muon,  pT, eta of electron \\
\hline
\verb!PythiaB.mhadr =  10. ;! &   number of repeated hadronizations  \\

&   select MHAD such that the average number of accepted
events with the same parton part is close to 1. The latter value
is printed out by PythiaB in the output log file.  \\
\hline
\verb!BSignalFilter.SignaltoNtup =  50.; ! &  For how many events store B-chains into NTUPLE  \\


\hline
\verb!PythiaB.SignalDumptoAscii = 0.; ! &  For how many events store exclusive signal B-chain into ascii fort.50 \\
\hline
\verb!PythiaB.flavour =  5.;! & wanted heavy flavour b(5) or c(4-not yet installed)\\
\hline




\end{tabular}




\subsection{  Data cards to define output files }

Data cards to define output files are  in the section {\bf Output files}.



\section{  Output files }

\subsection{NTUPLE file for B-decays chains}

\begin{verbatim}

//--------------------------------------------------------------
// NTuple output file
//--------------------------------------------------------------

NTupleSvc.Output = [ "FILE1
                      DATAFILE='pythiaB.ntup'
                      OPT='NEW'
                      TYP='HBOOK'" ]


\end{verbatim}


For each secondary particle from a B-signal and any other B-decay
chain in the event a following values in column-wise Ntuple are
written:

\begin{verbatim}
            "particles",
            "event_number",
            "chain_number",
            "particle_number",
            "id",
            "status",
            "child_of",

            "px",
            "py",
            "pz",
            "pe",
            "pt",
            "mass",

            "phi",
            "rapidity",
            "pseudorapidity"


\end{verbatim}




\subsection{B-signal chain, ascii file  }

Formated ascii file fort.50 containing lines each with one
particle from a B-signal chain.  Was used before BSignaFilter was
written. The file is created automatically if user defines non
zero values in PythiaB.SignalDumptoAscii = 10. and in
PythiaB.offcut = { 1., ...}.


Each line contains following information:
\begin{verbatim}

           float(ieve) ! this accepted ev number
           float(ITB)  ! # of evs hadronised to get this one
           float(ntree)! # of particles in B-chain
           float(itree)! this particle number in Bchain
           float(I)    ! this particle number in PYJET

           float(K(I,1))
           float(K(I,2))
           float(K(I,3))
           float(K(I,4))
           float(K(I,5))

           P(I,1)
           P(I,2)
           P(I,3)
           P(I,4)
           P(I,5)

           V(I,1)
           V(I,2)
           V(I,3)
           V(I,4)
           V(I,5)


\end{verbatim}



\subsection{Output log file}



Contains Pythia messages. List of all input datacards.
Pylist12 - list of all decay channels including status: opened-closed.
Cross section table. Summary of principle parameters and cuts.
Here is a part of log file:
\begin{verbatim}

 ==============================================================================
 I                                  I                            I            I
 I            Subprocess            I      Number of points      I    Sigma   I
 I                                  I                            I            I
 I----------------------------------I----------------------------I    (mb)    I
 I                                  I                            I            I
 I N:o Type                         I    Generated         Tried I            I
 I                                  I                            I            I
 ==============================================================================
 I                                  I                            I            I
 I   0 All included subprocesses    I         3095         32166 I  1.138E+00 I
 I  11 f + f' -> f + f' (QCD)       I           72           783 I  2.631E-02 I
 I  12 f + fbar -> f' + fbar'       I            3             6 I  3.266E-04 I
 I  13 f + fbar -> g + g            I            2             4 I  4.345E-04 I
 I  28 f + g -> f + g               I          818         10581 I  2.974E-01 I
 I  53 g + g -> f + fbar            I           67           262 I  2.491E-02 I
 I  68 g + g -> g + g               I         2133         20530 I  7.886E-01 I
 I                                  I                            I            I
 ==============================================================================

 ********* Fraction of events that fail fragmentation cuts =  0.00000 *********

I=====================================================================================
I   CROSSSECTION OF YOUR B-CHANNEL IS                   I
I                       BX= PX*NB/AC/MHAD=              I    2.77963e-05 mbarn
I                                                       I
I   IN CASE YOU FORCED ANY DECAY YOU SHOULD             I
I   CORRECT CROSS SECTION BX FURTHER, MULTIPLYING       I
I   BX BY BRANCHING RATIO(S) OF YOUR FORCED             I
I   DECAY(S) AND BY A FACTOR OF 2 FOR SYMMETRY          I
I                                                       I
I   MORE DETAILS ON CROSS SECTION                       I
I   PYTHIA MSEL=1     CROSS SECTION IS    PX=           I    1.13794 mbarn
I   NUMBER OF   ACCEPTED  MSEL=1 EVENTS   AC=           I    32166
I   NUMBER OF   ACCEPTED    B-EVENTS   IS NB=           I    11
I   REPEATED HADRONIZATIONS IN EACH EVENT MHAD=         I    14
I   AVERAGE NUM OF ACCEPTED EVTS IN HADRONIZATION LOOP  I    1.1
I   IN CASE YOU FORCED ANY DECAY YOU SHOULD             I
I   CORRECT CROSS SECTION BX FURTHER, MULTIPLYING       I
I   BX BY BRANCHING RATIO(S) OF YOUR FORCED             I
I   DECAY(S) AND BY A FACTOR OF 2 FOR SYMMETRY          I
I                                                       I
I=====================================================================================

I=====================================================================================
I          YOUR  MAIN SIMULATION PARAMETERS AND CUTS
I=====================================================================================
I   HARD SCATTERING  CUT  pysubs().ckin(3)             PT      I 15
I   STRUCTURE FCN (1=CTEQ3 7=CTEQ5) pypars().mstp(51)          I  1
I   CUTS ON b and/or anti b QUARK                              I  0 ; 102.5 ; and ; 10 ; 2.5
I   LVL1 MUON CUTS:                               PT AND ETA   I  6 ;  2.5
I   LVL2 CUTS:         ON(1)/OFF(0); PARTICLE-ID; PT AND ETA   I  0 ;  13 ;  6 ;  2.5
I   CUTS FOR STABLE PARTICLES IN B-DECAY:         ON(1)/OFF(0) I  1
I    CHARGED HADRONS:                             PT AND ETA   I  0.5 ;  2.5
I    MUONS:                                       PT AND ETA   I  3 ;  2.5
I    ELECTRONS:                                   PT AND ETA   I  0.5 ;  2.5
I=====================================================================================


\end{verbatim}


\subsection{AthenaRoot Persistency}

Produced Pythia B events are stored in root persistent file
PythiaB.root. This file  serves as an input to atlsim (detector simulation)
or atlfast.

\begin{verbatim}
//--------------------------------------------------------------
// AthenaRoot Persistency
//--------------------------------------------------------------

ApplicationMgr.DLLs += [ "EventHdrAthenaRoot", "GeneratorObjectsAthenaRoot"]
ApplicationMgr.DLLs += [ "RootSvcModules", "EventSelectorAthenaRoot", "AthenaRootCnvSvc"]
ApplicationMgr.ExtSvc += [ "RootSvc", "AthenaRootCnvSvc" ]
ApplicationMgr.DLLs += ["McEventSelector"]
ApplicationMgr.ExtSvc += [ "McCnvSvc", "McEventSelector/EventSelector" ]
EventPersistencySvc.CnvServices += [ "McCnvSvc" ]
ApplicationMgr.OutStream = [ "Stream1" ]
ApplicationMgr.OutStreamType = "AthenaOutputStream";
// for StoreGate
Stream1.ItemList = [ "2101#*", "133273#*"]
RootSvc.Output =["PythiaB.root"]
EventPersistencySvc.CnvServices += [ "AthenaRootCnvSvc" ]
Stream1.EvtConversionSvc = "AthenaRootCnvSvc"

\end{verbatim}

%\section{Bookeeping and cataloging}
\subsection{Which output  files we store in CASTOR}
B-physics group has a space in CASTOR
/afs/cern.ch/atlas/project/bphys/.
For password  into this area contact maria.smizanska@cern.ch.
Following files are to be stored:

\begin{itemize}
\vspace{-3mm}\item  PythiaB.root
\vspace{-3mm}\item  pythiaB.ntup
\vspace{-3mm}\item  output.log
\end{itemize}


%\subsection{Logical file names for Generator production.}

%Ouput files are recognized by their Logical File name defined
%as: \\
% \verb!LFN=$PROJECT.$DSET.$STEP.$PARTNR.$PGROUP.$DESCRIPTOR!.

%\begin{tabular}{|p{60mm}|p{100mm}|}
%\hline

%&    \\
% PROJECT  & =dc1 or dc2  \\ \hline
% DSET     & Number taken from a range of DSET numbers -
%            given to B-physics group.
%           Each institute participating in B-group will be given a subset
%         according to their request.
%          Contact maria.Smizanska@cern.ch     \\ \hline
% STEP     & =evgen  \\ \hline
% OUTPARTNR    &   one DSET have several OUTPARTNR assigned at atlsim simulation step.
%               e.q. 5000 Generator events with one DSET number will be simulated
%                  in atlsim in bunches of 500 events they will have 10
%              OUTPARTNR numbers \\ \hline
% PARTNR    &   one DSET have several OUTPARTNR assigned when sending jobs.
%               e.q. 5000 Generator events with one DSET number will be generated
%           in 5 bunches of 1000 events they will have 5
%               OUTPARTNR numbers \\ \hline

% PGROUP   & =Bphys \\ \hline
%DESCRIPTOR & e.q. \verb! Bd_Jpsi_mu6mu3_K0!  \
%                  \verb! bb_Jpsi_mumu_X !  \  \verb! Bs_Ds_phi_KK_pi ! \\
%          &    \\ \hline


%\end{tabular}


\section{List of all ATHENA files  dedicated to B-physics simulation  }
\begin{itemize}
\vspace{-3mm}\item PythiaB/src/PythiaB.cxx \vspace{-3mm}\item
PythiaB/src/$\rm PythiaB\_entries.cxx$
 \vspace{-3mm}\item
PythiaB/src/$\rm PythiaB\_load.cxx$ \vspace{-3mm}\item
PythiaB/src/*.F \vspace{-3mm}\item PythiaB/PythiaB/PythiaB.h

\vspace{-3mm}\item PythiaB/share/$\rm PythiaB\_Signal.py$
\vspace{-3mm}\item PythiaB/share/$\rm PythiaB\_bbmu6X.py$
\vspace{-3mm}\item PythiaB/share/Btune.py \vspace{-3mm}\item
PythiaB/share/CloseAntibQuark.py \vspace{-3mm}\item
PythiaB/share/Dsphipi.py

?


\vspace{12mm} \vspace{-3mm}\item
GeneratorFilters/src/BSignalFilter.cxx \vspace{-3mm}
\item
GeneratorFilters/GeneratorFilters/BSignalFilter.h

\end{itemize}

\section{Classes PythiaB inheriths from}
Pythia, GenModule

%\newpage
%\section{Structure of PythiaB code - Fig.}
%\begin{figure}[h]
%    \mbox{\epsfig{file=img2.eps,height=16.cm,width=17.cm}}
% \end{figure}

%\newpage
%\section{ PythiaB I$/$O - Fig.}
% \begin{figure}[h]
%    \mbox{\epsfig{file=img3.eps,height=16.cm,width=17.cm}}
% \end{figure}

\begin{thebibliography}{9}

\bibitem{cmt} ATLAS software developers guide, 
   \begin{verbatim} http://atlas-sw.cern.ch/cgi-bin/cvsweb.cgi/~checkout~/offline/ 
AtlasDoc/doc/SwDevUserGuide/userguide.html
\end{verbatim}
and cmt manual
\begin{verbatim} http://atlas.web.cern.ch/Atlas/GROUPS/SOFTWARE/OO/tools/cmt/Tutorial.html
\end{verbatim}
\bibitem{athena} Athena User Guide, 

\begin{verbatim} http://atlas.web.cern.ch/Atlas/GROUPS/SOFTWARE/OO/architecture/General/index.html
\end{verbatim}

\bibitem{Py6} T. Sjostrand et al,  PYTHIA 6.206 manual, LU TP 01-21 [hep-ph/0108264], 

\bibitem{Py6Nor} E.Norrbin, QCD phenomenology of Heavy particle dynamics, PhD theses, Lun University, Oct.2000.
\bibitem{my5}  S.P.Baranov, M.Smizanska, Phys.Rev.D62:014012,2000 and ATL-PHYS-98-133.
\bibitem{my6} Pythia6 Tuning for B-physics simulations in ATLAS.
NOTE under the preparation.

\end{thebibliography}
\end{document}
