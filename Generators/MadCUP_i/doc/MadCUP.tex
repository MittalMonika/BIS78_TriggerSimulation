\documentclass[11pt]{article}
\newdimen\SaveHeight \SaveHeight=\textheight
\textwidth=6.5in
\textheight=8.9in
\textwidth=6.5in
\textheight=9.0in
\hoffset=-.5in
\voffset=-1in
\def\topfraction{1.}
\def\textfraction{0.}   
\def\topfraction{1.}
\def\textfraction{0.}           
\begin{document}
\title{MadCUP\_i: An interface between MadCUP and Athena\\
Version in release 7.0.0 and later}
\author{  Georgios Stavropoulos (George.Stavropoulos@cern.ch) }

\maketitle           

This package runs MadCUP  from within Athena. \\See the example
in {\bf MadCUP\_i/share/jobOptions.MadCupPythia.py } and {\bf
  MadCUP\_i/share/jobOptions.MadCupHerwig.py }  which show how to
read MadCUP events and hadronize them using Pythia or Herwig

Users must first run 
MadCUP in standalone mode and make a file of events. An athena job
then takes these events hadronizes them and passes them down the
Athena event chain. The events must be made with a version of MadCUP
that is compatible, recent versions that support the Les Houches
interface should be acceptable.


To hadronize {\bf MadCUP} generated events with Herwig, you only need to run athena with the jobOptions
file jobOptions.MadCupHerwig.py by typing in the prompt \\
{\it athena jobOptions.MadCupHerwig.py}\\

To hadronize {\bf MadCUP} generated events with Pythia, you only need to run athena with the jobOptions
file jobOptions.MadCupPythia.py by typing in the prompt \\
{\it athena jobOptions.MadCupPythia.py}\\


More infomation about MadCUP  here

http://pheno.physics.wisc.edu/Software/MadCUP/

\end{document}







