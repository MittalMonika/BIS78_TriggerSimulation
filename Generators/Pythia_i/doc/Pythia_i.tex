\documentclass[11pt]{article}
\newdimen\SaveHeight \SaveHeight=\textheight
\textwidth=6.5in
\textheight=8.9in
\hoffset=-.5in
\voffset=-1in
\def\topfraction{1.}
\def\textfraction{0.}           

\begin{document}
\title{Pythia\_i: An interface between Pythia and Athena}
\author{ Ian Hinchliffe (I\_Hinchliffe@lbl.gov), G. Stavropoulos (George.Stavropoulos@cern.ch)}
%\today

\maketitle           

This package runs Pythia from within Athena, puts the events into the
transient store in HepMC format. See the documentation on GenModule  for
general information. The note refers only to Pythia specific
material. The External/Pythia package is used to set up the paths to
the Pythia  library which is now pointing to the release maintained by
Genser. The current Pythia version interfaced is 6.410

The module is activated from the jobOptions service. \\See the examples
in {\bf Pythia\_i/share/jobOptions.pythia.py }

The pythia parameters are set from the job options service.
{\bf Note that all parameters passed to Pythia are in the units
  specified in the Pythia manual. In particular, energies and
  masses are in GeV, not the standard atlas units of MeV.}
The following is needed if you wish to run Pythia
\begin{verbatim}
theAppDLLs  += [ "Pythia_i"]
theAppTopAlg = ["Pythia"]
\end{verbatim}
The parameters are passed via the following line in the jobOptions.py
file.

\begin{verbatim}Pythia.PythiaCommand = ["common_block_name variable index value", 
"common_block_name1 variable1 index1 value1"]
\end{verbatim} 

Each quoted string sets one parameter. You can have as many as you like
seperated by commas. 
{\bf common\_block\_name} must be one of the following common block names and
must be in lower case.\\
pydatr\\
pydat1\\
pydat2\\
pydat3\\
pypars\\
pymssm\\
pysubs\\
pyint2\\
pyinit\\
pystat\\
The pyinit deals with parameters associated with the job setup, while the pystat one deals
with the setting of the print level of the PYSTAT routine (called at the end of the run). 
An error message is returned if the common block is not one of
these. The job continues to run but the behaviour may not be what you
expect.
{\bf variable} must be the variable name that you are adjusting.
Valid names are as follows (lower case is required)\\
for pydatr -- mrpy and rrpy\\
for pydat1 -- mstu paru mstj and parj. Do not change mstu(11) (see below)\\
for pydat2 -- kchg pmas parf and vckm\\
for pydat3 -- mdcy mdme brat and kfdp\\
for pymssm -- imss and rmss\\
for pypars -- mstp parp msti and pari\\
for pysubs -- msel msub kfin and ckin\\
for pyint2 -- iset kfpr coef and icol\\
The following do not correspond to a common block but the parsing is
similar\\
pyinit  access variable that are either passed in the call to PYINIT
or to variables that control the listing. The choices are
pyinit -- pbar (changes one of the incoming particles to pbar), user
(changes the process selection to user code for connection to external
processes, a string then specifies the specific procees, see
discussion of external processes below), FIXT which switches to fixed
target operation, win
(changes the center of mass energy, units are GeV) output (controls
redirection of output) pylisti (gives
the number to be passed to pylist on the setup)  pylistf  (gives
the number to be passed to pylist on the dumped events) dumpr (is two
integers specifying the range of events to be written out).
%rndm\_IO
%(dumps/reads the random seeds used by PYTHIA to/from a file).\\
{\bf Settings in the pyinit groups should come before others. This
  is essential if you are using an external process such as comphep or
AcerMC}\\
pystat sets the print level of the PYSTAT routine (called at the end of the run).

Again an error message is returned if the common block is not one of
these. The job continues to run but the behaviour may not be what you
expect.

The remainder of the values in the " " specify the indices (if any) of
the variable and the value that you are setting it to. The range of
the indices is as described in the Pythia manual (Do not try to be
clever and offset them). You must specify the indices and value
completely. The number of quantities that you must provide depends on
the variable. The order is "(first\_index) (second\_index) value"; if
there is no corresponding index, omit it.
 There is currently minimal  error checking here so you will get junk or a 
core dump if you make a mistake. The variable and common block names are 
parsed and checked. A message is sent to the Athena LOG if an error is
detected but the job does not abort. 

Examples should make it clear\\

\noindent
"pysubs msel 13" will turn on Z/gamma+jet production\\
"pysubs ckin 3 18." will set the minimum $p_T$ to 18 GeV\\
"pypars mstp 43 2" wiil turn off the photon and Z/photon
interference.\\
"pyinit win 1800." changes the center of mass energy to 1800 GeV\\
Note that the entries are seperated by a single space and that reals
must have a decimal point.\\
"pyinit pylisti 12" dumps all the particle and decay data (see pythia
manual)  after initialization.\\
"pyinit output junk.dat" causes all the pythia output to dump into a
file junk.dat (it resets mstu(11))\\
"pyinit  pylistf 1 " dumps the complete event record for the specified
events\\
"pyinit dumpr 3 12 " causes events 6 through 12 to be written out\\
``pystat 3 5 7'' sets the PYSTAT print levels. Pystat is called as many times
as integers after pystat. In the example, pystat will be called with prin-level 3 followed by
a call with print-level 5 and then by a call with print-level 7. The default is to call pystat
once with print-level 1.\\
"pydat2 pmas 4000011 1 1000." will set the mass of particle with {\bf KF code} 4000011 to 1000 GeV.
{\bf WARNING: Someone has to give the KF and NOT the KC code. The KF code is converted (via a call to
PYCOMP) to the KC code internally in Pythia.cxx}\\

% The format of the rndm\_IO option of the pyinit ``menu'' is the following\\
% "pyinit rndm\_IO file\_name Switch FirstEvent SkipEvents MOVE"\\
% Where\\
% {\bf file\_name } is the name of the file to/from where PYTHIA dumps/reads the rndm seeds\\
% {\bf Switch } Control variable : 0 (default)=NO action,
% 1=Write (pyrget is called, paragarph 4.3 of PYTHIA manual), 2=Read (pyrset is called).An Athena
% INFO message is printed for every event generated with rndm seeds read from the file.\\
% {\bf FirstEvent } For Switch=1 is the first Event to be dumped in the file. For Switch=2
% is the first record in the file to be read. Default value is 1. Values $<$ 1 are not allowed
% and the Switch flag is set to 0 internally by the program (an Athena error message is printed)\\
% {\bf SkipEvents } For Switch=1 is the number of events to be skipped before the next
% event to be dumped in the file. For Switch=2 is the number of records to be skipped before
% the next record is read. Default value is 0. Values $<$ 0 are not allowed
% and the Switch flag is set to 0 internally by the program (an Athena error message is printed)\\
% {\bf MOVE } Sets the MOVE (2nd argument) of the pyrget and pyrset subroutines of PYTHIA
% (Paragarph 4.3 of PYTHIA manual). Default value is 0. Setting this variable to a
% value different than 0 when Switch=2, ignores FirstEvent and SkipEvents.

The jobOptions.tex file that has this example is contained in {\bf Pythia\_i/share/jobOptions.pythia.py }\\


{\large \bf Default Parameters}\\

The ATLAS settings are set inside Pythia.cxx. The default initialization corresponds to $pp$ collsions at 14 TeV, PYLIST is called 
as PYLIST(11) after initialization.
The current set of defaults is the ones in the Pythia release 6.410 with the 
following exceptions.
\begin{itemize}
\item pmas(6,1)=175.;  // top quark mass
\item pmas(24,1)=80.42; // W mass
\item pmas(24,2)=2.124; // W width
\item pmas(23,1)=91.19; // Z0 mass
\item pmas(23,2)=2.495; // Z0 width
\item mstp(128)=1;     // fix junk output for documentary particles
\item mstu(21)=1;     // error handling switch
\item mstp(81)=21; // ATLAS default for MI and also for the associated
treatment of initial- and final-state showers and beam remnants. MI on; new model for PYEVNW.
\item mstp(70)=2;      // (virtuality scale for ISR)
\item mstp(72)=0;
\item mstp(82)=4;      // (mutiple interaction model)
\item mstp(84)=1;
\item mstp(85)=1;
\item mstp(86)=2;
\item mstp(87)=4;
\item mstp(88)=0;
\item mstp(89)=1;
\item mstp(90)=1;
\item mstp(95)=1;
\item parp(78)=0.2;
\item parp(80)=0.01;
\item parp(82)=1.9;    // (cut off scale)
\item parp(83)=0.3;
\item parp(84)=0.5;    // (matter distribution)
\item parp(89)=1800.;
\item parp(90)=0.22;   // (cut off scale)
\item mstj(11)=3;      // (select peterson for charm fragmentation)
\item mstj(22)=2;      // (Make K\_S, Lambda stable)
\item parj(54)=-0.07;  // (c hadronization)
\item parj(55)=-0.006; // (b hadronization)
\item parj(81)=0.14;
\item mstp(52)=2;      // (needed for CTEQ6L)
\item mstp(54)=2;      // (needed for CTEQ6L)
\item mstp(56)=2;      // (needed for CTEQ6L)
\item mstp(51)=10042;  // (needed for CTEQ6L)
\item mstp(53)=10042;  // (needed for CTEQ6L)
\item mstp(55)=10042;  // (needed for CTEQ6L)
\end{itemize}


{\large \bf Random Numbers}\\

 Pythia.cxx is using the AtRndmGenSvc Athena Service to provide to Pythia (via the pyr function,
 found in Pythia\_i/src/pyr.F) the necessary random numbers. This service is using the RanecuEngine of CLHEP,
 and is based on the ``stream'' logic, each stream being able to provide an idependent sequence of random
 numbers. Pythia.cxx is using two streams: PYTHIA\_INIT and PYTHIA. The first stream is used to provide
 random numbers for the initialization phase of Pythia and the second one for the event generation. The user
 can set the initial seeds of each stream via the following option in the jobOption file.

 \begin{verbatim} 
 AtRndmGenSvc.Seeds = [``PYTHIA_INIT 2345533 9922199'', ``PYTHIA 5498921 659091'']
 \end{verbatim}

 The above sets the seeds of the PYTHIA\_INIT stream to 2345533 and 9922199 and of the PYTHIA one to
 5498921 and 659091. If the user will not set the seeds of a stream then the AtRndmGenSvc will use default
 values.

 The seeds of the Random number service are saved for each event in the HepMC Event record and they are printed
 on screen by DumpMC. In this way an event can be reproduced easily. The user has to rerun the job by simply seting
 the seeds of the PYTHIA stream (the seeds of the PYTHIA\_INIT stream should stay the same) to the seeds of that
 event.

 Additionaly the AtRndmGenSvc is dumping into a file (AtRndmGenSvc.out) the seeds of all the streams at the end
 of the job. This file can be read back by the service if the user set the option
 \begin{verbatim} AtRndmGenSvc.ReadFromFile = True \end{verbatim} (default = False). In this case the file
 AtRndmGenSvc.out is read and the streams saved in this file are created with seeds as in this file. The name of
 the file to be read can be set by the user via the option
 \begin{verbatim} AtRndmGenSvc.FileToRead = MyFileName \end{verbatim}

 The above file is also written out when a job crashes.
 
 The {\bf Pythia\_i/share/jobOptions.pythia.py } contains the above options.\\
 
 {\large \bf User modifications}\\

Two types of user modifications are common

\begin{itemize}
\item If you are trying to replace an existing routine that is in the
  Pythia library this is straightforward.
Assume that you are trying to replace test.f that exists in Pythia.
Check out Pythia\_i under CMT,  (use cmt co -r Pythia\_i-xx-xx-xx  Generators/Pythia\_i where
xx-xx-xx is the version in the release that you are running against), put your version of test.f into the /src area of the checked out
code. Then in the /cmt area edit the requirements file and add test.f
into the list of files that get complied. Note that each generator has
its own library. You must therefore put your file in the right
place. For Pythia, here is the example.
\item If you want to access ``External Process''. This is done in
  Pythia by setting ``USER''(see section 9.9 of the 6.4 Pythia
  manual). External procesess usually read a file containting events.  At present several externals are available,
  CompHep, AcerMC, Alpgen, Matchig, HvGen, MadCUP, MadGraph, TopRex, GR@PPA,
  and the fourth is a dummy that users can adapt to their
  needs by following one of the examples above. To find out how to run these external processes
  please refer to the documentation of these generators. \\
  To add your own external, create your own  inituser.f and useuser.f,
  put these filenames into the cmt/requirements so that they are built
  into libPythia\_i.so and then rebuild the Pythia\_i  package  (cmt make) and then
  set\\ 
  "pyinit user user''

{\large \bf Note on LHApdf stucture functions}

In the case you want to run Pythia with the LHAPDF structure functions you need to
set the mstp(52), mstp(54) and mstp(56) variables equal to 2 and the mstp(51),
mstp(53) and mstp(55) ones to the LHAPDF set/member index (see the documentation
of the Generators/Lhapdf\_i package for the LHAPDF set/member index settings).

\end{itemize}



\end{document}







