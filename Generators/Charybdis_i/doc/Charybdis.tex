\documentclass[11pt]{article}
\newdimen\SaveHeight \SaveHeight=\textheight
\textwidth=6.5in
\textheight=8.9in
\textwidth=6.5in
\textheight=9.0in
\hoffset=-.5in
\voffset=-1in
\def\topfraction{1.}
\def\textfraction{0.}   
\def\topfraction{1.}
\def\textfraction{0.}  
         
\title{Chayrbdis\_i: An interface between the Charybdis and Athena }
\author{ Nick Brett (n.brett1@physics.ox.ac.uk) }

\begin{document}

\thispagestyle{empty}

\maketitle           

\section*{Introduction}

This pacakge runs the Charybdis Micro Black Hole Monte Carlo Generator
from within Athena. The Charybdis generator has been configured to use Herwig for
hadronisation and interacts with it via the Les Houches accord.

\section*{Available Input Parameters}

Charybdis has a number of user defined input parameters which can be used to alter its behavior. Some but not all of these parameters can be modified through a jobOptions file using the Athena interface to Charybdis. Those that are available are listed bellow.

\subsection*{Plank Mass}

This is the fundamental Plank mass as experienced by any physics on length scales smaller than the size of the extra dimensions. Current collider limits place the fundamental Plank mass above ~800 GeV. If extra dimensions are to solve the hierarchy problem then the fundamental Plank mass would ideally be on the order of 1 TeV.

\subsection*{Total Number of Dimensions}

This is the total number of large dimensions including the (3+1) dimensions that are apparent in current physics.

\subsection*{Black Hole Mass}

Micro Black Holes produced through particles collisions will have a range of masses defined by the range of center of mass energies of those collisions.

\subsection*{Number of Particles in Remnant Decay}

Micro Black Holes may be considered to follow the normal rules of
General Relativity and Hawking radiation provided that their mass is
more than a few multiples of the fundamental Plank mass. As a Micro
Black Hole decays via Hawking radiation its mass falls until
eventually ( $\sim 10^{-26}$ seconds later !) it can no longer be considered
as a "classical" object. At this point some new unknown physics must
begin to define the Black Holes behavior. The Black Hole However, will
still have some mass and possibly carry some other conserved
quantities such as electric charge. The Charybdis Monte Carlo program
deals with this remaining object (a Remnant of a black hole) by
causing it to spontaneously decay into a user defined number of
Standard model particles. This option sets the number of those particles.

\subsection*{Time Variation of Black Hole Temperature}

A Black Holes Hawking temperature if defined by its mass and the
number of dimensions in which it exists. As a Black Hole evaporates
(by emitting particles via the Hawking process) its mass falls and so
its temperature increases. However, because Micro Black Holes have a
very high Hawking temperature the time between particle emissions is
extremely short. This leads to the question of whether or not a Micro
Black Hole has time to reach thermal equilibrium in between each
particle emission. When this option is set to true the Charybdis
assumes that the Black Hole decays in quasi-static equilibrium and
its temperature is always a function of its immediate mass. When it is
false the Black Holes temperature remains constant throughout its decay.

\subsection*{Grey Body Factors}

To first order a Black Hole behaves like a perfect black body thermal
emitter. However, the gravitational field surrounding a Black Hole as
well as any charge or angular momentum it may have define a set of
potentials in its vicinity. Particles emitted by the black hole
interact with this potential in different ways depending on their
charge, spin etc. As a result the spectra of emitted particles is
modified, transforming the black body spectrum into what is know as a
Grey body spectrum. Setting this option true causes Charybdis to take
account of these Grey body factors.

\subsection*{Kinematic Cut}

This parameter determines whether Charybdis conserves energy and
momentum as it decays. When set true any generated decays which exceed
the kinematic limit will be disregarded and replaced by a newly
generated decay.

\section*{How to Write a JobOptions File}

All the parameters detailed above can be configured through a
JobOptions file using the Herwig Command string. Commands consist of
two integers proceeded by the phrase "charyb", as in the following example:

\begin{verbatim}
Herwig.HerwigCommand += [ "charyb 2 6" ]     # Total number of dimensions
\end{verbatim}

The phrase "charyb" indicates that the command should be interpreted by the Charybdis interface rather than any other part of Herwig\_i. The first of the two integers indicates which parameter is to be modified and the second provides the value which should be assigned to the parameter.

Each of the parameters has a default value which will be passed to Charybdis unless a modified value is defined by the user.
Table of Parameters

\subsection*{Command Summary}

The table bellow lists all available parameters along with the relevant number and the default values:

\begin{tabular}{ l c c }
Option &	Number &	Default\\
\hline 
Plank Mass &	1 &	1000 (GeV)\\
Total number of dimensions &	2 &	8\\
Minimum Black Hole Mass &	3 &	5000 (GeV)\\
Maximum Black Hole Mass &	4 &	14000 (GeV)\\
Number of particles in remnant decay &	5 &	2\\
Time variation of Black hole temperature &	6 & 	1 (true)\\
Grey Body factors &	7 &	1 (true)\\
Kinematic cut enabled &	8 &	0 (false)\\
\end{tabular}

\subsection*{Example Configuration}

Bellow is an example of a complete Charybdis configuration. Herwig is selected as a generator and provided with random number seeds as usual, then options are passed to Charybdis as described above:

\begin{verbatim}
AtRndmGenSvc = Service( "AtRndmGenSvc" )
AtRndmGenSvc.Seeds =["HERWIG 1804289383 846930886", "HERWIG\_INIT 1681692777 1714636915"];

theApp.TopAlg += ["Herwig"]

Herwig = Algorithm( "Herwig" )
Herwig.OutputLevel = INFO
Herwig.HerwigCommand =  [ "iproc charybdis" ]     # Set Herwig to run charybdis

Herwig.HerwigCommand += [ "charyb 2 6" ]     # Total number of dimensions
Herwig.HerwigCommand += [ "charyb 3 6000" ]  # Min Black Hole mass
Herwig.HerwigCommand += [ "charyb 4 6500" ]  # Max Black Hole mass
Herwig.HerwigCommand += [ "charyb 6 0" ]     # Time Variation of mass
Herwig.HerwigCommand += [ "charyb 7 0" ]     # Grey body factors
\end{verbatim}

\section*{Further Reading}
\begin{description}
\item[ATLAS TWiki page] https://twiki.cern.ch/twiki/bin/view/AtlasProtected/CharybdisForAtlas
\item[Charybdis Authors website] http://www.ippp.dur.ac.uk/montecarlo/leshouches/generators/charybdis/
\end{description}

\end{document}
